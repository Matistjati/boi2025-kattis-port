\problemname{Gingerbread}
Toruń has been known for its traditional gingerbread since the Middle Ages. Young Nicolaus would like
to buy a set of $n$ boxes with gingerbread cookies in his favourite shop. The shop has very strict rules,
though: Nicolaus initially obtains $n$ boxes that are already filled with cookies:
the $i$-th box initially contains $a_{i}$ of them. Then, Nicolaus can order some extra cookies.
He adds extra cookies to some boxes so that the greatest common divisor\footnote{
The greatest common divisor (GCD) of multiple numbers is the largest positive integer
that divides all of them without remainder.}
of the numbers of cookies in all of the boxes becomes equal to 1.
It can be proven that this is always possible.

Help Nicolaus by calculating the smallest number of cookies that need to be added in order
to make the greatest common divisor of all the numbers equal to 1.

\section*{Input}
The first line contains an integer $n$ ($2\leq n\leq 10^{6}$),
denoting the number of boxes.

The second line contains $n$ integers $a_{1}, a_{2}, \ldots, a_{n}$ ($1 \leq a_i \leq 10^{7}$),
where the $i$-th integer $a_{i}$ denotes the initial number of cookies in the $i$-th box.

\section*{Output}
Output one line with a single integer denoting the smallest number of cookies that Nicolaus
should add to the boxes.
If Nicolaus doesn't have to add any cookies to make the greatest common divisor
of the numbers equal to 1, output 0.

\section*{Scoring}
\begin{center}
\begin{tabular}{|c|p{13cm}|c|}
\hline
\textbf{Subtask} & \textbf{Constraints} & \textbf{Points} \\
\hline
1 & $n = 2$ & 17 \\\hline
2 & $n \leq 10$ & 34 \\\hline
3 & $n \leq 1000$ & 11 \\\hline
4 & No additional constraints. & 38\\\hline
\end{tabular}
\end{center}

\section*{Explanation of sample case 1}
Indeed, the greatest common divisor (GCD) of 90, 84, and 140 is 2, so some cookies must be added.
If we add only one cookie, we may obtain the quantities 91, 84, 140 with GCD of 7,
or 90, 85, 140 with GCD of 5, or 90, 84, 141 with GCD of 3, so this is not enough.
After adding two cookies, one to the first box, and one to the second box, we obtain the quantities 91, 85, 140 with GCD of 1; hence the answer is 2.
Note that adding both cookies to the first box does not help: we obtain quantities 92, 84, 140 with GCD of 4.
